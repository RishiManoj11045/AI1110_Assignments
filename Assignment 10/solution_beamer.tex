%%%%%%%%%%%%%%%%%%%%%%%%%%%%%%%%%%%%%%%%%%%%%%%%%%%%%%%%%%%%%%%
%
% Welcome to Overleaf --- just edit your LaTeX on the left,
% and we'll compile it for you on the right. If you open the
% 'Share' menu, you can invite other users to edit at the same
% time. See www.overleaf.com/learn for more info. Enjoy!
%
%%%%%%%%%%%%%%%%%%%%%%%%%%%%%%%%%%%%%%%%%%%%%%%%%%%%%%%%%%%%%%%


% Inbuilt themes in beamer
\documentclass{beamer}

\def\inputGnumericTable{}

\usepackage[latin1]{inputenc}                                 %%
\usepackage{color}                                            %%
\usepackage{array}                                            %%
\usepackage{longtable}                                        %%
\usepackage{calc}                                             %%
\usepackage{multirow}                                         %%
\usepackage{hhline}                                           %%
\usepackage{ifthen}                                           %%


% Theme choice:
\usetheme{CambridgeUS}

% Title page details: 
\title{Assignment 10 \\ Papoullis Textbook Chapter 9 Ex 9.30} 
\author{Rishi Manoj - CS21BTECH11045}
\date{\today}
\logo{\large \LaTeX{}}

\providecommand{\cbrak}[1]{\ensuremath{\left\{#1\right\}}}

\begin{document}

% Title page frame
\begin{frame}
    \titlepage 
\end{frame}

% Remove logo from the next slides
\logo{}


% Outline frame
\begin{frame}{Outline}
    \tableofcontents
\end{frame}


\section{Question}
\begin{frame}{Question}
The input of a linear system with $h(t) = Ae^{-\alpha t}U(t)$ is a process of $x(t)$ with $R_x(\tau)=N\delta(\tau)$ applied at $t=0$ and disconnected at $t=T$. Find $E\cbrak{y^2(t)}$.
\end{frame}


\section{Solution}
\begin{frame}{Solution}
Given, $h(t) = Ae^{-\alpha t}U(t)$, $R_x(\tau)=N\delta(\tau)$, applied at $t=0$ and disconnected at $t=T$. Also $q(t)=N$ for $0<t<T$ and 0 otherwise.
\end{frame}


\section{For $0<t<T$}
\begin{frame}{For $0<t<T$}
For $0<t<T$, $E\cbrak{y^2(t)}$ is given as,
	\begin{align}
	       E\cbrak{y^2(t)} &= N\int_{0}^{t}h^2(\tau)d\tau \\
	                       &= NA^2\int_{0}^{t}e^{-2\alpha\tau}d\tau \\
	                       &= \frac{NA^2}{2\alpha}(1-e^{-2\alpha t})
    \end{align}
\end{frame}


\section{For $t\ge T$}
\begin{frame}{For $t\ge T$}
For $t\ge T$,given $q(t)=0$. So, $E\cbrak{y^2(t)}$ is given as,
    \begin{align}
           E\cbrak{y^2(t)} &= q(t)\int_{0}^{t}h^2(\tau)d\tau \\
	                       &= q(t)\int_{0}^{T}h^2(\tau)d\tau + q(t)\int_{T}^{t}h^2(\tau)d\tau \\
	                       &= NA^2\int_{0}^{T}e^{-2\alpha\tau}d\tau + 0 \\ 
	                       &= \frac{NA^2}{2\alpha}(1-e^{-2\alpha T})
    \end{align}
\end{frame}


\section{$E\cbrak{y^2(t)}$}
\begin{frame}{$E\cbrak{y^2(t)}$}
In the above cases $U(t)$ is taken as 1 as t is positive.
    Therefore,
    \begin{equation*}
           E\cbrak{y^2(t)} = \begin{cases}
                              \frac{NA^2}{2\alpha}(1-e^{-2\alpha t}),  & 0<t<T \\
                              \frac{NA^2}{2\alpha}(1-e^{-2\alpha T}),  & T\le t
                             \end{cases}
    \end{equation*}
\end{frame}

\end{document}