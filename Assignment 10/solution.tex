\documentclass[journal,12pt,twocolumn]{IEEEtran}
%
\usepackage{setspace}
\usepackage{gensymb}
%\doublespacing
\singlespacing

%\usepackage{graphicx}
%\usepackage{amssymb}
%\usepackage{relsize}
\usepackage[cmex10]{amsmath}
%\usepackage{amsthm}
%\interdisplaylinepenalty=2500
%\savesymbol{iint}
%\usepackage{txfonts}
%\restoresymbol{TXF}{iint}
%\usepackage{wasysym}
\usepackage{amsthm}
\usepackage{amssymb}
\usepackage{cite}
\usepackage{cases}
\usepackage{subfig}
%\usepackage{xtab}
\usepackage{longtable}
\usepackage{multirow}
%\usepackage{algorithm}
%\usepackage{algpseudocode}
\usepackage{booktabs}
\usepackage{enumitem}
\usepackage{mathtools}
\usepackage{tikz}
\usepackage{pgfplots}
\usepackage{circuitikz}
\usepackage{verbatim}
\usepackage{tfrupee}
\usepackage[breaklinks=true]{hyperref}
%\usepackage{stmaryrd}
\usepackage{tkz-euclide} % loads  TikZ and tkz-base
%\usetkzobj{all}
\usetikzlibrary{fit}
\usetikzlibrary{calc,math}
%\pgfdeclarelayer{background}
%\pgfsetlayers{background}
\usepackage{listings}
\usepackage{color}                                            %%
\usepackage{array}                                            %%
\usepackage{longtable}                                        %%
\usepackage{calc}                                             %%
\usepackage{multirow}                                         %%
\usepackage{hhline}                                           %%
\usepackage{ifthen}                                           %%
%optionally (for landscape tables embedded in another document): %%
\usepackage{lscape}     
\usepackage{multicol}
\usepackage{chngcntr}
%\usepackage{enumerate}

%\usepackage{wasysym}
%\newcounter{MYtempeqncnt}
\DeclareMathOperator*{\Res}{Res}
%\renewcommand{\baselinestretch}{2}
\renewcommand\thesection{\arabic{section}}
\renewcommand\thesubsection{\thesection.\arabic{subsection}}
\renewcommand\thesubsubsection{\thesubsection.\arabic{subsubsection}}

\renewcommand\thesectiondis{\arabic{section}}
\renewcommand\thesubsectiondis{\thesectiondis.\arabic{subsection}}
\renewcommand\thesubsubsectiondis{\thesubsectiondis.\arabic{subsubsection}}

% correct bad hyphenation here
\hyphenation{op-tical net-works semi-conduc-tor}
\def\inputGnumericTable{}                                 %%

\lstset{
	%language=C,
	frame=single, 
	breaklines=true,
	columns=fullflexible
}
%\lstset{
	%language=tex,
	%frame=single, 
	%breaklines=true
	%}

\begin{document}
	%
	
	\bibliographystyle{IEEEtran}
	%\bibliographystyle{ieeetr}
	\providecommand{\mbf}{\mathbf}
	\providecommand{\pr}[1]{\ensuremath{\Pr\left(#1\right)}}
	\providecommand{\qfunc}[1]{\ensuremath{Q\left(#1\right)}}
	\providecommand{\sbrak}[1]{\ensuremath{{}\left[#1\right]}}
	\providecommand{\lsbrak}[1]{\ensuremath{{}\left[#1\right.}}
	\providecommand{\rsbrak}[1]{\ensuremath{{}\left.#1\right]}}
	\providecommand{\brak}[1]{\ensuremath{\left(#1\right)}}
	\providecommand{\lbrak}[1]{\ensuremath{\left(#1\right.}}
	\providecommand{\rbrak}[1]{\ensuremath{\left.#1\right)}}
	\providecommand{\cbrak}[1]{\ensuremath{\left\{#1\right\}}}
	\providecommand{\lcbrak}[1]{\ensuremath{\left\{#1\right.}}
	\providecommand{\rcbrak}[1]{\ensuremath{\left.#1\right\}}}
	\providecommand{\dec}[2]{\ensuremath{\overset{#1}{\underset{#2}{\gtrless}}}}
	\newcommand{\myvec}[1]{\ensuremath{\begin{pmatrix}#1\end{pmatrix}}}
	\newcommand{\mydet}[1]{\ensuremath{\begin{vmatrix}#1\end{vmatrix}}}
	\newcommand*{\permcomb}[4][0mu]{{{}^{#3}\mkern#1#2_{#4}}}
	\newcommand*{\perm}[1][-3mu]{\permcomb[#1]{P}}
	\newcommand*{\comb}[1][-1mu]{\permcomb[#1]{C}}
		\title{
				AI1110: Assignment 10
		}
		\author{
			Rishi Manoj - CS21BTECH11045
		}
			
	\maketitle
	\begin{abstract}
		This document contains the solution to Question of Chapter 9 in the Papoullis Textbook.
	\end{abstract}
	
	\textbf{Chapter 9 Ex 9.30:}
	The input of a linear system with $h(t) = Ae^{-\alpha t}U(t)$ is a process of $x(t)$ with $R_x(\tau)=N\delta(\tau)$ applied at $t=0$ and disconnected at $t=T$. Find $E\cbrak{y^2(t)}$. 
	
	\textbf{Solution:}
	Given, $h(t) = Ae^{-\alpha t}U(t)$, $R_x(\tau)=N\delta(\tau)$, applied at $t=0$ and disconnected at $t=T$. Also $q(t)=N$ for $0<t<T$ and 0 otherwise. For $0<t<T$, $E\cbrak{y^2(t)}$ is given as,
	\begin{align}
	       E\cbrak{y^2(t)} &= N\int_{0}^{t}h^2(\tau)d\tau \\
	                       &= NA^2\int_{0}^{t}e^{-2\alpha\tau}d\tau \\
	                       &= \frac{NA^2}{2\alpha}(1-e^{-2\alpha t})
    \end{align}
    For $t\ge T$,given $q(t)=0$. So, $E\cbrak{y^2(t)}$ is given as,
    \begin{align}
           E\cbrak{y^2(t)} &= q(t)\int_{0}^{t}h^2(\tau)d\tau \\
	                       &= q(t)\int_{0}^{T}h^2(\tau)d\tau + q(t)\int_{T}^{t}h^2(\tau)d\tau \\
	                       &= NA^2\int_{0}^{T}e^{-2\alpha\tau}d\tau + 0 \\ 
	                       &= \frac{NA^2}{2\alpha}(1-e^{-2\alpha T})
    \end{align}
    In the above cases $U(t)$ is taken as 1 as t is positive.
    Therefore,
    \begin{equation*}
           E\cbrak{y^2(t)} = \begin{cases}
                              \frac{NA^2}{2\alpha}(1-e^{-2\alpha t}),  & 0<t<T \\
                              \frac{NA^2}{2\alpha}(1-e^{-2\alpha T}),  & T\le t
                             \end{cases}
    \end{equation*}
\end{document}