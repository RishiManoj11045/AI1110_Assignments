\documentclass[journal,12pt,twocolumn]{IEEEtran}
%
\usepackage{setspace}
\usepackage{gensymb}
%\doublespacing
\singlespacing

%\usepackage{graphicx}
%\usepackage{amssymb}
%\usepackage{relsize}
\usepackage[cmex10]{amsmath}
%\usepackage{amsthm}
%\interdisplaylinepenalty=2500
%\savesymbol{iint}
%\usepackage{txfonts}
%\restoresymbol{TXF}{iint}
%\usepackage{wasysym}
\usepackage{amsthm}
\usepackage{cite}
\usepackage{cases}
\usepackage{subfig}
%\usepackage{xtab}
\usepackage{longtable}
\usepackage{multirow}
%\usepackage{algorithm}
%\usepackage{algpseudocode}
\usepackage{booktabs}
\usepackage{enumitem}
\usepackage{mathtools}
\usepackage{tikz}
\usepackage{pgfplots}
\usepackage{circuitikz}
\usepackage{verbatim}
\usepackage{tfrupee}
\usepackage[breaklinks=true]{hyperref}
%\usepackage{stmaryrd}
\usepackage{tkz-euclide} % loads  TikZ and tkz-base
%\usetkzobj{all}
\usetikzlibrary{fit}
\usetikzlibrary{calc,math}
%\pgfdeclarelayer{background}
%\pgfsetlayers{background}
\usepackage{listings}
\usepackage{color}                                            %%
\usepackage{array}                                            %%
\usepackage{longtable}                                        %%
\usepackage{calc}                                             %%
\usepackage{multirow}                                         %%
\usepackage{hhline}                                           %%
\usepackage{ifthen}                                           %%
%optionally (for landscape tables embedded in another document): %%
\usepackage{lscape}     
\usepackage{multicol}
\usepackage{chngcntr}
%\usepackage{enumerate}

%\usepackage{wasysym}
%\newcounter{MYtempeqncnt}
\DeclareMathOperator*{\Res}{Res}
%\renewcommand{\baselinestretch}{2}
\renewcommand\thesection{\arabic{section}}
\renewcommand\thesubsection{\thesection.\arabic{subsection}}
\renewcommand\thesubsubsection{\thesubsection.\arabic{subsubsection}}

\renewcommand\thesectiondis{\arabic{section}}
\renewcommand\thesubsectiondis{\thesectiondis.\arabic{subsection}}
\renewcommand\thesubsubsectiondis{\thesubsectiondis.\arabic{subsubsection}}

% correct bad hyphenation here
\hyphenation{op-tical net-works semi-conduc-tor}
\def\inputGnumericTable{}                                 %%

\lstset{
	%language=C,
	frame=single, 
	breaklines=true,
	columns=fullflexible
}
%\lstset{
	%language=tex,
	%frame=single, 
	%breaklines=true
	%}

\begin{document}
	%
	
	\bibliographystyle{IEEEtran}
	%\bibliographystyle{ieeetr}
	\providecommand{\mbf}{\mathbf}
	\providecommand{\pr}[1]{\ensuremath{\Pr\left(#1\right)}}
	\providecommand{\qfunc}[1]{\ensuremath{Q\left(#1\right)}}
	\providecommand{\sbrak}[1]{\ensuremath{{}\left[#1\right]}}
	\providecommand{\lsbrak}[1]{\ensuremath{{}\left[#1\right.}}
	\providecommand{\rsbrak}[1]{\ensuremath{{}\left.#1\right]}}
	\providecommand{\brak}[1]{\ensuremath{\left(#1\right)}}
	\providecommand{\lbrak}[1]{\ensuremath{\left(#1\right.}}
	\providecommand{\rbrak}[1]{\ensuremath{\left.#1\right)}}
	\providecommand{\cbrak}[1]{\ensuremath{\left\{#1\right\}}}
	\providecommand{\lcbrak}[1]{\ensuremath{\left\{#1\right.}}
	\providecommand{\rcbrak}[1]{\ensuremath{\left.#1\right\}}}
	\providecommand{\dec}[2]{\ensuremath{\overset{#1}{\underset{#2}{\gtrless}}}}
	\newcommand{\myvec}[1]{\ensuremath{\begin{pmatrix}#1\end{pmatrix}}}
	\newcommand{\mydet}[1]{\ensuremath{\begin{vmatrix}#1\end{vmatrix}}}
	\newcommand*{\permcomb}[4][0mu]{{{}^{#3}\mkern#1#2_{#4}}}
	\newcommand*{\perm}[1][-3mu]{\permcomb[#1]{P}}
	\newcommand*{\comb}[1][-1mu]{\permcomb[#1]{C}}
		\title{
				AI1110: Assignment 4
		}
		\author{
			Rishi Manoj - CS21BTECH11045
		}
			
	\maketitle
    \begin{abstract}
		This document contains the solution to Question of Chapter 16 (Probability) in the NCERT Class 11 Textbook.
    \end{abstract}
	
    \textbf{Probability Miscellaneous Exercise Q2:}
	4 cards are drawn from a well-shuffled deck of 52 cards. What is the probability of obtaining 3 diamonds and one spade?
	
	\textbf{Solution:}
	Let $X$ and $Y$ be the events defined as follows,\\
	$X$ : Picking three diamond cards \\
	$Y$ : Picking a spade card \\
	\begin{align}
	n(X) &= {13 \choose 3} \\
	n(Y) &= {13 \choose 1} \\
	n(X Y) &= {13 \choose 3} \times {13 \choose 1} \\
	n(S) &= {52 \choose 4}
	\end{align}
	Here, $S$ is the sample space.\\
	Probability of getting 3 diamonds and 1 spade is given by,
	\begin{align}
	\pr{X Y} &= \frac{n(X Y)}{n(S)} \\
	&= \frac{{13 \choose 3} \times {13 \choose 1}}{{52 \choose 4}} \\
	&= \frac{3178}{270725} \\
	&= 0.0137
	\end{align}
	Therefore, the probability of getting 3 diamonds and 1 spade is 0.0137.
\end{document}
	