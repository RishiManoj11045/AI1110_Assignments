%%%%%%%%%%%%%%%%%%%%%%%%%%%%%%%%%%%%%%%%%%%%%%%%%%%%%%%%%%%%%%%
%
% Welcome to Overleaf --- just edit your LaTeX on the left,
% and we'll compile it for you on the right. If you open the
% 'Share' menu, you can invite other users to edit at the same
% time. See www.overleaf.com/learn for more info. Enjoy!
%
%%%%%%%%%%%%%%%%%%%%%%%%%%%%%%%%%%%%%%%%%%%%%%%%%%%%%%%%%%%%%%%


% Inbuilt themes in beamer
\documentclass{beamer}

\def\inputGnumericTable{}

\usepackage[latin1]{inputenc}                                 %%
\usepackage{color}                                            %%
\usepackage{array}                                            %%
\usepackage{longtable}                                        %%
\usepackage{calc}                                             %%
\usepackage{multirow}                                         %%
\usepackage{hhline}                                           %%
\usepackage{ifthen}                                           %%


% Theme choice:
\usetheme{CambridgeUS}

% Title page details: 
\title{Assignment 9 \\ Papoullis Textbook Chapter 8 Ex 8.14} 
\author{Rishi Manoj - CS21BTECH11045}
\date{\today}
\logo{\large \LaTeX{}}

\providecommand{\brak}[1]{\ensuremath{\left(#1\right)}}

\begin{document}

% Title page frame
\begin{frame}
    \titlepage 
\end{frame}

% Remove logo from the next slides
\logo{}


% Outline frame
\begin{frame}{Outline}
    \tableofcontents
\end{frame}


\section{Question}
\begin{frame}{Question}
A coin is tossed once, and heads shows. Assuming that the probability $p$ of heads is the value of the random variable \textbf{p} uniformly distributed in the interval (0.4,0.6), find the bayesian estimate.
\end{frame}


\section{Solution}
\begin{frame}{Solution}
Given that the probability is uniformly distributed in the interval (0.4,0.6). So, in this interval,
	\begin{align}
	          f(p)(0.6-0.4) &= 1 \\
	          f(p) &= 5
    \end{align}
\end{frame}


\section{$f(p)$}
\begin{frame}{$f(p)$}
Therefore, $f(p)$ is defined in the following way,
    \begin{equation*}
           f(p)  = \begin{cases}
                    5,  & 0.4<p<0.6 \\
                    0,  & otherwise
                   \end{cases}
    \end{equation*}
\end{frame}


\section{$\hat{p}$}
\begin{frame}{$\hat{p}$}
The estimate $\hat{p}$ is given by,
    \begin{align}
           \hat{p} &= \int_{0.4}^{0.6} pf(p)dp \\
                   &= 5\int_{0.4}^{0.6} pdp \\
                   &= 0.5
    \end{align} 
\end{frame}


\section{Solving}
\begin{frame}{Solving}
The posterior density $f(p|M)$ in the interval (0.4,0.6), when $M$ is 1 (because here it is only one head), is given by,
    \begin{align}
           f(p|1) &= \frac{pf(p)}{\int_{0.4}^{0.6}pf(p)dp} \\
                  &= \frac{(p)(5)}{0.5} \\
                  &= 10p
    \end{align}
\end{frame}


\section{$f(p|1)$}
\begin{frame}{$f(p|1)$}
Therefore, $f(p|1)$ is defined in the following way,
    \begin{equation*}
           f(p|1)  = \begin{cases}
                    10p,  & 0.4<p<0.6 \\
                    0,  & otherwise
                   \end{cases}
    \end{equation*}
\end{frame}


\section{Updated $\hat{p}$}
\begin{frame}{Updated $\hat{p}$}
The updated estimate $\hat{p}$ of $p$ is the conditional estimate of \textbf{p} assuming $M$, which is given by,
    \begin{align}
           \hat{p} &= \int_{0.4}^{0.6} pf(p|1)dp \\
                   &= 10\int_{0.4}^{0.6} p^2dp \\
                   &= 0.5067
    \end{align}
\end{frame}

\end{document}