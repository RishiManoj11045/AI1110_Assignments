\documentclass[journal,12pt,twocolumn]{IEEEtran}
\usepackage{polynom}
\usepackage{amsmath}
\title{AI1110 Assignment-1\\ICSE Class-10 2017}
\author{Rishi Manoj\\CS21BTECH11045}
\begin{document}
\maketitle

\providecommand{\brak}[1]{\ensuremath{\left(#1\right)}}

\textbf{Q4(a):}
What must be subtracted from $16x^3-8x^2+4x+7$ so that the resulting expression has $2x+1$ as a factor?\\
\textbf{Solution:}
Let $p(x)=16x^3-8x^2+4x+7$ and $d(x)=2x+1$.\\
Polynomial division of $p(x)$ with $d(x)$:\\
\\
$\polylongdiv{16x^3-8x^2+4x+7}{2x+1}$
\\
\\
From the above division, it is clear that 1 has to be subtracted from the polynomial $p(x)$, so that $d(x)$ becomes factor of the resulting polynomial after subtraction.
\\
\\
Using remainder theorem:\\
Remainder theorem states that remainder of division of a polynomial $f(x)$ by a linear polynomial $x-r$ is equal to $f(r)$.So, remainder of $p(x)$ divided by $d(x)$ is $p\brak{\frac{-1}{2}}$.

\begin{align}
      p(x) \mod\ (2x+1) &= p\brak{\frac{-1}{2}}
\end{align}

\begin{align}
      p\brak{\frac{-1}{2}} &= 16\brak{\frac{-1}{2}}^3-8\brak{\frac{-1}{2}}^2+4\brak{\frac{-1}{2}}+7
      \\
      &= 1
\end{align}
Therefor, the remainder is:
\begin{align}
      p(x) \mod\ (2x+1) &= 1
\end{align}

So, subtracting 1 from the given polynomial $p(x)$ gives a polynomial which has $2x+1$ as its factor.
\end{document}