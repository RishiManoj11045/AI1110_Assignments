\let\negmedspace\undefined
\let\negthickspace\undefined
%\RequirePackage{amsmath}
\documentclass[journal,12pt,twocolumn]{IEEEtran}
 \usepackage[utf8]{inputenc}
 \usepackage{graphicx}
 \usepackage{amsmath}
 \usepackage{mathrsfs}
\usepackage{txfonts}
\usepackage{stfloats}
\usepackage{bm}
\usepackage{cite}
\usepackage{cases}
\usepackage{subfig}
 \usepackage{amsfonts}
 \usepackage{amssymb}
 \usepackage{enumitem}
\usepackage{mathtools}
\usepackage{tikz}
\usepackage{circuitikz}
\usepackage{verbatim}
\usepackage[breaklinks=false,hidelinks]{hyperref}
\usepackage{listings}
\usepackage{calc}
\usepackage{float}
\usepackage{longtable}
\usepackage{multirow}
\usepackage{multicol}
\usepackage{color}
\usepackage{array}
\usepackage{hhline}
\usepackage{ifthen}
\usepackage{chngcntr}

\newcommand{\BEQA}{\begin{eqnarray}}
\newcommand{\EEQA}{\end{eqnarray}}
\newcommand{\define}{\stackrel{\triangle}{=}}
\bibliographystyle{IEEEtran}
%\bibliographystyle{ieeetr}
\def\inputGnumericTable{}
\let\vec\mathbf
\providecommand{\pr}[1]{\ensuremath{\Pr\left(#1\right)}}
\providecommand{\sbrak}[1]{\ensuremath{{}\left[#1\right]}}
\providecommand{\lsbrak}[1]{\ensuremath{{}\left[#1\right.}}
\providecommand{\rsbrak}[1]{\ensuremath{{}\left.#1\right]}}
\providecommand{\brak}[1]{\ensuremath{\left(#1\right)}}
\providecommand{\lbrak}[1]{\ensuremath{\left(#1\right.}}
\providecommand{\rbrak}[1]{\ensuremath{\left.#1\right)}}
\providecommand{\cbrak}[1]{\ensuremath{\left\{#1\right\}}}
\providecommand{\lcbrak}[1]{\ensuremath{\left\{#1\right.}}
\providecommand{\rcbrak}[1]{\ensuremath{\left.#1\right\}}}
\providecommand{\abs}[1]{\left\vert#1\right\vert}
\providecommand{\res}[1]{\Res\displaylimits_{#1}}
\newcommand{\myvec}[1]{\ensuremath{\begin{pmatrix}#1\end{pmatrix}}}
\newcommand{\mydet}[1]{\ensuremath{\begin{vmatrix}#1\end{vmatrix}}}
\newcommand{\solution}{\noindent \textbf{Solution: }}
\title{Assignment 3}
\author{Rishi Manoj\\CS21BTECH11045}
\date{}
\begin{document}
% make the title area
\maketitle
\begin{abstract}
		This document contains the solution to Question of Chapter 15 (Probability) in the NCERT Class 10 Textbook.
\end{abstract}
	
	\textbf{Probability Excercise 15.1 Q17.}
	 
\begin{enumerate}[label=(\roman{enumi})]
     \item A lot of 20 bulbs contain 4 defective ones. One bulb is drawn at  random fronthe lot. What is the probability that this bulb is defective?
     \item Suppose the bulb drawn above is not defective and is not replaced. Now one bulb is drawn at random from the rest. What is the probability that this bulb is not defective?
\end{enumerate} 
\solution Let $X\in \cbrak{0,1}$ is a random variable that denotes whether the bulb drawn is defective or not.
\begin{table}[ht!]
		\centering
		\input{Tables/table1.tex}
		\caption{Random Variable and Event Distribution}
		\label{table:1}
\end{table}
\begin{enumerate}[label=(\roman{enumi})]
     \item The probability that the drawn bulb is defective can be given as:
     \begin{align}
              \pr{X=0} &= \frac{\text{Number of defective bulbs}}{\text{Total number of bulbs}}\\
              &= \frac{4}{20}\\
              &= 0.2
     \end{align}
     \item The bulb drawn in the above case is supposed to be not defective and is not replaced. Now, a bulb is drawn at random from the lot.\\
     The probability that the drawn bulb is not defective can be given as:
     \begin{align}
              \pr{X=1} &= \frac{\text{Number of non-defective bulbs}}{\text{Total number of bulbs}}\\
              &= \frac{15}{19}\\
              &= 0.789
     \end{align}
\end{enumerate}
\end{document} 