%%%%%%%%%%%%%%%%%%%%%%%%%%%%%%%%%%%%%%%%%%%%%%%%%%%%%%%%%%%%%%%
%
% Welcome to Overleaf --- just edit your LaTeX on the left,
% and we'll compile it for you on the right. If you open the
% 'Share' menu, you can invite other users to edit at the same
% time. See www.overleaf.com/learn for more info. Enjoy!
%
%%%%%%%%%%%%%%%%%%%%%%%%%%%%%%%%%%%%%%%%%%%%%%%%%%%%%%%%%%%%%%%


% Inbuilt themes in beamer
\documentclass{beamer}

\def\inputGnumericTable{}

\usepackage[latin1]{inputenc}                                 %%
\usepackage{color}                                            %%
\usepackage{array}                                            %%
\usepackage{longtable}                                        %%
\usepackage{calc}                                             %%
\usepackage{multirow}                                         %%
\usepackage{hhline}                                           %%
\usepackage{ifthen}                                           %%


% Theme choice:
\usetheme{CambridgeUS}

% Title page details: 
\title{Assignment 7 \\ Papoullis Textbook Chapter 6 Example 13} 
\author{Rishi Manoj - CS21BTECH11045}
\date{\today}
\logo{\large \LaTeX{}}

\providecommand{\cbrak}[1]{\ensuremath{\left\{#1\right\}}}

\begin{document}

% Title page frame
\begin{frame}
    \titlepage 
\end{frame}

% Remove logo from the next slides
\logo{}


% Outline frame
\begin{frame}{Outline}
    \tableofcontents
\end{frame}


\section{Question}
\begin{frame}{Question}
Let $z = x^2 + y^2$. Determine $f_z(z)$.
\end{frame}


\section{Solution}
\begin{frame}{Solution}
$F_z(z)$ can be defined in the following way,
	\begin{align}
	       F_z(z) &= P\cbrak{x^2+y^2 \le z} \\
	              &= \int \int_{x^2+y^2 \le z}f_{xy}(x,y)dx dy
	\end{align}
\end{frame}


\section{Solving}
\begin{frame}{Solving}
Here, $x^2+y^2 \le z$ represents the area of a circle and the radius of the circle is $\sqrt{z}$. Therefore, 
	\begin{align}
	      F_z(z) &= \int_{y=-\sqrt{z}}^{\sqrt{z}} \int_{x=-\sqrt{z-y^2}}^{\sqrt{z-y^2}}f_{xy}(x,y)dx dy 
    \end{align}
\end{frame}


\section{$f_z(z)$}
\begin{frame}{$f_z(z)$}
From the above equation we get,
    \begin{align}
          f_z(z) &= \int_{-\sqrt{z}}^{\sqrt{z}}\frac{1}{2\sqrt{z-y^2}}\cbrak{f_{xy}(\sqrt{z-y^2},y)+f_{xy}(-\sqrt{z-y^2},y)}dy
    \end{align}
\end{frame}

\end{document}