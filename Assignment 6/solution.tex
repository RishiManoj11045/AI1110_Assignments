\documentclass[journal,12pt,twocolumn]{IEEEtran}
%
\usepackage{setspace}
\usepackage{gensymb}
%\doublespacing
\singlespacing

%\usepackage{graphicx}
%\usepackage{amssymb}
%\usepackage{relsize}
\usepackage[cmex10]{amsmath}
%\usepackage{amsthm}
%\interdisplaylinepenalty=2500
%\savesymbol{iint}
%\usepackage{txfonts}
%\restoresymbol{TXF}{iint}
%\usepackage{wasysym}
\usepackage{amsthm}
\usepackage{amssymb}
\usepackage{cite}
\usepackage{cases}
\usepackage{subfig}
%\usepackage{xtab}
\usepackage{longtable}
\usepackage{multirow}
%\usepackage{algorithm}
%\usepackage{algpseudocode}
\usepackage{booktabs}
\usepackage{enumitem}
\usepackage{mathtools}
\usepackage{tikz}
\usepackage{pgfplots}
\usepackage{circuitikz}
\usepackage{verbatim}
\usepackage{tfrupee}
\usepackage[breaklinks=true]{hyperref}
%\usepackage{stmaryrd}
\usepackage{tkz-euclide} % loads  TikZ and tkz-base
%\usetkzobj{all}
\usetikzlibrary{fit}
\usetikzlibrary{calc,math}
%\pgfdeclarelayer{background}
%\pgfsetlayers{background}
\usepackage{listings}
\usepackage{color}                                            %%
\usepackage{array}                                            %%
\usepackage{longtable}                                        %%
\usepackage{calc}                                             %%
\usepackage{multirow}                                         %%
\usepackage{hhline}                                           %%
\usepackage{ifthen}                                           %%
%optionally (for landscape tables embedded in another document): %%
\usepackage{lscape}     
\usepackage{multicol}
\usepackage{chngcntr}
%\usepackage{enumerate}

%\usepackage{wasysym}
%\newcounter{MYtempeqncnt}
\DeclareMathOperator*{\Res}{Res}
%\renewcommand{\baselinestretch}{2}
\renewcommand\thesection{\arabic{section}}
\renewcommand\thesubsection{\thesection.\arabic{subsection}}
\renewcommand\thesubsubsection{\thesubsection.\arabic{subsubsection}}

\renewcommand\thesectiondis{\arabic{section}}
\renewcommand\thesubsectiondis{\thesectiondis.\arabic{subsection}}
\renewcommand\thesubsubsectiondis{\thesubsectiondis.\arabic{subsubsection}}

% correct bad hyphenation here
\hyphenation{op-tical net-works semi-conduc-tor}
\def\inputGnumericTable{}                                 %%

\lstset{
	%language=C,
	frame=single, 
	breaklines=true,
	columns=fullflexible
}
%\lstset{
	%language=tex,
	%frame=single, 
	%breaklines=true
	%}

\begin{document}
	%
	
	\bibliographystyle{IEEEtran}
	%\bibliographystyle{ieeetr}
	\providecommand{\mbf}{\mathbf}
	\providecommand{\pr}[1]{\ensuremath{\Pr\left(#1\right)}}
	\providecommand{\qfunc}[1]{\ensuremath{Q\left(#1\right)}}
	\providecommand{\sbrak}[1]{\ensuremath{{}\left[#1\right]}}
	\providecommand{\lsbrak}[1]{\ensuremath{{}\left[#1\right.}}
	\providecommand{\rsbrak}[1]{\ensuremath{{}\left.#1\right]}}
	\providecommand{\brak}[1]{\ensuremath{\left(#1\right)}}
	\providecommand{\lbrak}[1]{\ensuremath{\left(#1\right.}}
	\providecommand{\rbrak}[1]{\ensuremath{\left.#1\right)}}
	\providecommand{\cbrak}[1]{\ensuremath{\left\{#1\right\}}}
	\providecommand{\lcbrak}[1]{\ensuremath{\left\{#1\right.}}
	\providecommand{\rcbrak}[1]{\ensuremath{\left.#1\right\}}}
	\providecommand{\dec}[2]{\ensuremath{\overset{#1}{\underset{#2}{\gtrless}}}}
	\newcommand{\myvec}[1]{\ensuremath{\begin{pmatrix}#1\end{pmatrix}}}
	\newcommand{\mydet}[1]{\ensuremath{\begin{vmatrix}#1\end{vmatrix}}}
	\newcommand*{\permcomb}[4][0mu]{{{}^{#3}\mkern#1#2_{#4}}}
	\newcommand*{\perm}[1][-3mu]{\permcomb[#1]{P}}
	\newcommand*{\comb}[1][-1mu]{\permcomb[#1]{C}}
		\title{
				AI1110: Assignment 6
		}
		\author{
			Rishi Manoj - CS21BTECH11045
		}
			
	\maketitle
	\begin{abstract}
		This document contains the solution to Question of Chapter 5 in the Papoullis Textbook.
	\end{abstract}
	
	\textbf{Chapter 5 Ex 33:}
	A person writes $n$ letters and addresses $n$ envelopes. Then one letter is randomly placed into each envelope. What is the probability that at least one letter will reach its correct destination? 
	
	\textbf{Solution:}
	Let $X_k$ represents the event that there are exactly $k$ coincidences among the $n$ envelopes. All the events corresponding to the values of $k$ starting from 0 to $n$ are mutually exclusive and exhaustive. So, by theorem of total probability,
	\begin{align}
	       & p_n(0)+p_n(1)+p_n(2)+\cdots+p_n(n) = 1 &\\
	       & p_n(k) \triangleq P(X_k) &
	\end{align}
	Number of ways of drawing $k$ letters from a group of $n$ is ${n \choose k}$.
	Probability of generating $k$ coincidences with the chosen $k$ letters is,
	\begin{align}
	        P &= \brak{\frac{1}{n}}\brak{\frac{1}{n-1}}\cdots\brak{\frac{1}{n-k+1}} 
	\end{align}
	Probability of no coincidences with remaining $n-k$ letters is given by $p_{n-k}(0)$.
	So,
	\begin{align}
	 p_n(k) &= {n \choose k}\brak{\frac{1}{n(n-1)\cdots(n-k+1)}}p_{n-k}(0) \\
	 &= \frac{p_{n-k}(0)}{k\!}
	\end{align}
	We know that,
	\begin{align}
	           p_n(n) &= \frac{1}{n!}
	\end{align}
	Substituiting eq.(5) in eq.(1) along with the above one gives,
	\begin{align}
	 & p_n(0)+\frac{p_{n-1}(0)}{1!}+\cdots+\frac{p_{1}(0)}{(n-1)!}+\frac{1}{n!} = 1 &
	\end{align}
	From this we can get the following,
	\begin{align}
	          p_1(0) &= 0 \\
	          p_2(0) &= \frac{1}{2} \\
	          p_3(0) &= \frac{1}{6}
	\end{align}
	We define the moment generating function in the following way to get an explicit expression for $p_n(0)$,
	\begin{align}
	         \phi(z) &= \Sigma^\infty_{n=0}p_n(0)z^n \\
	         e^z\phi(z) &= \brak{\Sigma^\infty_{k=0}\frac{z^k}{k!}}\brak{\Sigma^\infty_{n=0}p_n(0)z^n} \\
	         &= 1+z+z^2+\cdots+z^n+\cdots \\
	         &= \frac{1}{1-z}
	\end{align}
	Using the above equations, we get,
	\begin{align}
	\phi(z) &= \frac{e^{-z}}{1-z} \\
	        &= \Sigma^\infty_{n=0}\brak{\Sigma^n_{k=0}\frac{(-1)^k}{k!}}z^n \\
	p_n(0) &= \Sigma^n_{k=0}\frac{(-1)^k}{k!} \rightarrow \frac{1}{e} \\
	       &= 0.378
	\end{align} 
	Using eq.(5) we get,
	\begin{align}
	         p_n(k) &= \frac{1}{k!}\Sigma^{n-k}_{m=0}\frac{(-1)^m}{m!}
	\end{align}
	The required probability is given by,
	\begin{align}
	   Pr &= 1-p_n(0) \\
	   &= 1-\Sigma^n_{k=0}\frac{(-1)^k}{k!} \rightarrow 0.632	
	\end{align}      
	           
\end{document}
	
	