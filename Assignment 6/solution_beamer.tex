%%%%%%%%%%%%%%%%%%%%%%%%%%%%%%%%%%%%%%%%%%%%%%%%%%%%%%%%%%%%%%%
%
% Welcome to Overleaf --- just edit your LaTeX on the left,
% and we'll compile it for you on the right. If you open the
% 'Share' menu, you can invite other users to edit at the same
% time. See www.overleaf.com/learn for more info. Enjoy!
%
%%%%%%%%%%%%%%%%%%%%%%%%%%%%%%%%%%%%%%%%%%%%%%%%%%%%%%%%%%%%%%%


% Inbuilt themes in beamer
\documentclass{beamer}

\def\inputGnumericTable{}

\usepackage[latin1]{inputenc}                                 %%
\usepackage{color}                                            %%
\usepackage{array}                                            %%
\usepackage{longtable}                                        %%
\usepackage{calc}                                             %%
\usepackage{multirow}                                         %%
\usepackage{hhline}                                           %%
\usepackage{ifthen}                                           %%


% Theme choice:
\usetheme{CambridgeUS}

% Title page details: 
\title{Assignment 6 \\ Papoullis Textbook Chapter 5 Example 33} 
\author{Rishi Manoj - CS21BTECH11045}
\date{\today}
\logo{\large \LaTeX{}}

\providecommand{\brak}[1]{\ensuremath{\left(#1\right)}}

\begin{document}

% Title page frame
\begin{frame}
    \titlepage 
\end{frame}

% Remove logo from the next slides
\logo{}


% Outline frame
\begin{frame}{Outline}
    \tableofcontents
\end{frame}


\section{Question}
\begin{frame}{Question}
A person writes $n$ letters and addresses $n$ envelopes. Then one letter is randomly placed into each envelope. What is the probability that at least one letter will reach its correct destination?
\end{frame}


\section{Solution}
\begin{frame}{Solution}
Let $X_k$ represents the event that there are exactly $k$ coincidences among the $n$ envelopes. All the events corresponding to the values of $k$ starting from 0 to $n$ are mutually exclusive and exhaustive. So, by theorem of total probability,
	\begin{align}
	       & p_n(0)+p_n(1)+p_n(2)+\cdots+p_n(n) = 1 &\\
	       & p_n(k) \triangleq P(X_k) &
	\end{align}
Let $P$ be the probability of generating $k$ coincidences with the chosen $k$ letters and $Pr$ be the required probability that atleast one letter reaches its correct destination.
\end{frame}


\section{Probability $P$}
\begin{frame}{Probability $P$}
Number of ways of drawing $k$ letters from a group of $n$ is ${n \choose k}$.
	Probability of generating $k$ coincidences with the chosen $k$ letters is,
	\begin{align}
	        P &= \brak{\frac{1}{n}}\brak{\frac{1}{n-1}}\cdots\brak{\frac{1}{n-k+1}} 
	\end{align}
\end{frame}


\section{Probability $p_n(k)$}
\begin{frame}{Probability $p_n(k)$}
Probability of no coincidences with remaining $n-k$ letters is given by $p_{n-k}(0)$.
	So,
	\begin{align}
	 p_n(k) &= {n \choose k}\brak{\frac{1}{n(n-1)\cdots(n-k+1)}}p_{n-k}(0) \\
	 &= \frac{p_{n-k}(0)}{k\!}
	\end{align}
\end{frame}


\section{Solving}
\begin{frame}{Solving}
We know that,
	\begin{align}
	           p_n(n) &= \frac{1}{n!}
	\end{align}
	Substituiting eq.(5) in eq.(1) along with the above one gives,
	\begin{align}
	 & p_n(0)+\frac{p_{n-1}(0)}{1!}+\cdots+\frac{p_{1}(0)}{(n-1)!}+\frac{1}{n!} = 1 &
	\end{align}
	From this we can get the following,
	\begin{align}
	          p_1(0) &= 0 \\
	          p_2(0) &= \frac{1}{2} \\
	          p_3(0) &= \frac{1}{6}
	\end{align}
\end{frame}


\section{Moment Generating Function}
\begin{frame}{Moment Generating Function}
We define the moment generating function in the following way to get an explicit expression for $p_n(0)$,
	\begin{align}
	         \phi(z) &= \Sigma^\infty_{n=0}p_n(0)z^n \\
	         e^z\phi(z) &= \brak{\Sigma^\infty_{k=0}\frac{z^k}{k!}}\brak{\Sigma^\infty_{n=0}p_n(0)z^n} \\
	         &= 1+z+z^2+\cdots+z^n+\cdots \\
	         &= \frac{1}{1-z}
	\end{align}
\end{frame}


\section{Solving for $p_n(0)$}
\begin{frame}{Solving for $p_n(0)$}
Using the above equations, we get,
	\begin{align}
	\phi(z) &= \frac{e^{-z}}{1-z} \\
	        &= \Sigma^\infty_{n=0}\brak{\Sigma^n_{k=0}\frac{(-1)^k}{k!}}z^n \\
	p_n(0) &= \Sigma^n_{k=0}\frac{(-1)^k}{k!} \rightarrow \frac{1}{e} \\
	       &= 0.378
	\end{align} 
	Using eq.(5) we get,
	\begin{align}
	         p_n(k) &= \frac{1}{k!}\Sigma^{n-k}_{m=0}\frac{(-1)^m}{m!}
	\end{align}
\end{frame}


\section{Probability $Pr$}
\begin{frame}{Probability $Pr$}
The required probability is given by,
	\begin{align}
	   Pr &= 1-p_n(0) \\
	   &= 1-\Sigma^n_{k=0}\frac{(-1)^k}{k!} \rightarrow 0.632	
	\end{align}
\end{frame}

\end{document}
