\documentclass[journal,12pt,twocolumn]{IEEEtran}
%
\usepackage{setspace}
\usepackage{gensymb}
%\doublespacing
\singlespacing

%\usepackage{graphicx}
%\usepackage{amssymb}
%\usepackage{relsize}
\usepackage[cmex10]{amsmath}
%\usepackage{amsthm}
%\interdisplaylinepenalty=2500
%\savesymbol{iint}
%\usepackage{txfonts}
%\restoresymbol{TXF}{iint}
%\usepackage{wasysym}
\usepackage{amsthm}
\usepackage{amssymb}
\usepackage{cite}
\usepackage{cases}
\usepackage{subfig}
%\usepackage{xtab}
\usepackage{longtable}
\usepackage{multirow}
%\usepackage{algorithm}
%\usepackage{algpseudocode}
\usepackage{booktabs}
\usepackage{enumitem}
\usepackage{mathtools}
\usepackage{tikz}
\usepackage{pgfplots}
\usepackage{circuitikz}
\usepackage{verbatim}
\usepackage{tfrupee}
\usepackage[breaklinks=true]{hyperref}
%\usepackage{stmaryrd}
\usepackage{tkz-euclide} % loads  TikZ and tkz-base
%\usetkzobj{all}
\usetikzlibrary{fit}
\usetikzlibrary{calc,math}
%\pgfdeclarelayer{background}
%\pgfsetlayers{background}
\usepackage{listings}
\usepackage{color}                                            %%
\usepackage{array}                                            %%
\usepackage{longtable}                                        %%
\usepackage{calc}                                             %%
\usepackage{multirow}                                         %%
\usepackage{hhline}                                           %%
\usepackage{ifthen}                                           %%
%optionally (for landscape tables embedded in another document): %%
\usepackage{lscape}     
\usepackage{multicol}
\usepackage{chngcntr}
%\usepackage{enumerate}

%\usepackage{wasysym}
%\newcounter{MYtempeqncnt}
\DeclareMathOperator*{\Res}{Res}
%\renewcommand{\baselinestretch}{2}
\renewcommand\thesection{\arabic{section}}
\renewcommand\thesubsection{\thesection.\arabic{subsection}}
\renewcommand\thesubsubsection{\thesubsection.\arabic{subsubsection}}

\renewcommand\thesectiondis{\arabic{section}}
\renewcommand\thesubsectiondis{\thesectiondis.\arabic{subsection}}
\renewcommand\thesubsubsectiondis{\thesubsectiondis.\arabic{subsubsection}}

% correct bad hyphenation here
\hyphenation{op-tical net-works semi-conduc-tor}
\def\inputGnumericTable{}                                 %%

\lstset{
	%language=C,
	frame=single, 
	breaklines=true,
	columns=fullflexible
}
%\lstset{
	%language=tex,
	%frame=single, 
	%breaklines=true
	%}

\begin{document}
	%
	
	\bibliographystyle{IEEEtran}
	%\bibliographystyle{ieeetr}
	\providecommand{\mbf}{\mathbf}
	\providecommand{\pr}[1]{\ensuremath{\Pr\left(#1\right)}}
	\providecommand{\qfunc}[1]{\ensuremath{Q\left(#1\right)}}
	\providecommand{\sbrak}[1]{\ensuremath{{}\left[#1\right]}}
	\providecommand{\lsbrak}[1]{\ensuremath{{}\left[#1\right.}}
	\providecommand{\rsbrak}[1]{\ensuremath{{}\left.#1\right]}}
	\providecommand{\brak}[1]{\ensuremath{\left(#1\right)}}
	\providecommand{\lbrak}[1]{\ensuremath{\left(#1\right.}}
	\providecommand{\rbrak}[1]{\ensuremath{\left.#1\right)}}
	\providecommand{\cbrak}[1]{\ensuremath{\left\{#1\right\}}}
	\providecommand{\lcbrak}[1]{\ensuremath{\left\{#1\right.}}
	\providecommand{\rcbrak}[1]{\ensuremath{\left.#1\right\}}}
	\providecommand{\dec}[2]{\ensuremath{\overset{#1}{\underset{#2}{\gtrless}}}}
	\newcommand{\myvec}[1]{\ensuremath{\begin{pmatrix}#1\end{pmatrix}}}
	\newcommand{\mydet}[1]{\ensuremath{\begin{vmatrix}#1\end{vmatrix}}}
	\newcommand*{\permcomb}[4][0mu]{{{}^{#3}\mkern#1#2_{#4}}}
	\newcommand*{\perm}[1][-3mu]{\permcomb[#1]{P}}
	\newcommand*{\comb}[1][-1mu]{\permcomb[#1]{C}}
		\title{
				AI1110: Assignment 8
		}
		\author{
			Rishi Manoj - CS21BTECH11045
		}
			
	\maketitle
	\begin{abstract}
		This document contains the solution to Question of Chapter 6 in the Papoullis Textbook.
	\end{abstract}
	
	\textbf{Chapter 6 Ex 6.70:}
	The random variables $x$ and $y$ are $N(3,4,1,4,0.5)$. Find $f(y|x)$ and $f(x|y)$.
	
	\textbf{Solution:}
	Given $N(3,4,1,4,0.5)$.
	\begin{align}
	       \mu_1 &= 3 \\
	       \mu_2 &= 4 \\
	       \sigma_1^2 &= 1 \\ 
	       \sigma_2^2 &= 1 \\ 	       
	       r &= 0.5
	\end{align}
	$f(y|x)$ is given by,
	\begin{align}
	        f(y|x) &= \frac{1}{\sigma _2\sqrt{2\pi(1-r^2)}}exp\brak{\frac{-((y-\eta_2)-r\sigma_2(x-\eta_1)/\sigma_1)^2}{2\sigma_2^2(1-r^2)}}
	\end{align}
	Substituiting the corresponding values and simplifying gives,
	\begin{align}
	        f(y|x) &= \frac{1}{2\sqrt{2\pi(1-(0.5)^2)}}exp\brak{\frac{-((y-4)-(0.5)(2)(x-3)/1)^2}{2(4)(1-(0.5)^2)}} \\
	        f(y|x) &= \frac{1}{\sqrt{6\pi}}exp\brak{\frac{-(y-x-1)^2}{6}}
	\end{align}
	$f(x|y)$ is given by,
	\begin{align}
	        f(x|y) &= \frac{1}{\sigma _1\sqrt{2\pi(1-r^2)}}exp\brak{\frac{-((x-\eta_1)-r\sigma_1(y-\eta_2)/\sigma_2)^2}{2\sigma_1^2(1-r^2)}}
	\end{align}
	Substituiting the corresponding values and simplifying gives,
	\begin{align}
	        f(y|x) &= \frac{1}{1\sqrt{2\pi(1-(0.5)^2)}}exp\brak{\frac{-((x-3)-(0.5)(1)(y-4)/1)^2}{2(1)(1-(0.5)^2)}} \\
	        f(y|x) &= \frac{\sqrt{2}}{\sqrt{3\pi}}exp\brak{\frac{-(4y-x-13)^2}{24}}
	\end{align}
\end{document}